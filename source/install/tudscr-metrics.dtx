% \iffalse meta-comment
%
%  TUD-Script -- Corporate Design of Technische Universität Dresden
% ----------------------------------------------------------------------------
%
%  Copyright (C) Falk Hanisch <hanisch.latex@outlook.com>, 2012-2021
%
% ----------------------------------------------------------------------------
%
%  This work may be distributed and/or modified under the conditions of the
%  LaTeX Project Public License, either version 1.3c of this license or
%  any later version. The latest version of this license is in
%    http://www.latex-project.org/lppl.txt
%  and version 1.3c or later is part of all distributions of
%  LaTeX version 2008-05-04 or later.
%
%  This work has the LPPL maintenance status "maintained".
%
%  The current maintainer and author of this work is Falk Hanisch.
%
% ----------------------------------------------------------------------------
%
% \fi
%
% \iffalse ins:batch + dtx:driver
%<*ins>
\ifx\documentclass\undefined
  \input docstrip.tex
  \keepsilent
  \askforoverwritefalse
  \generate{%
    \nopreamble\nopostamble%
    \file{installfonts.tex}{\from{tudscr-metrics.dtx}{install}}%
    \file{createmap.tex}{\from{tudscr-metrics.dtx}{map}}%
    \file{ligature.mtx}{\from{tudscr-metrics.dtx}{metric,general,ligature}}%
    \file{unsetgreek.mtx}{\from{tudscr-metrics.dtx}{metric,general,greek}}%
    \file{setoldnum.mtx}{\from{tudscr-metrics.dtx}{metric,general,numbers}}%
    \file{lunot1.mtx}{\from{tudscr-metrics.dtx}{metric,univers,OT1}}%
    \file{lunt1.mtx}{\from{tudscr-metrics.dtx}{metric,univers,T1}}%
    \file{lunts1.mtx}{\from{tudscr-metrics.dtx}{metric,univers,TS1}}%
    \file{lunoml.mtx}{\from{tudscr-metrics.dtx}{metric,univers,OML}}%
    \file{lunoms.mtx}{\from{tudscr-metrics.dtx}{metric,univers,OMS}}%
    \file{lunsymb.mtx}{\from{tudscr-metrics.dtx}{metric,univers,symbols}}%
    \file{0m6ot1.mtx}{\from{tudscr-metrics.dtx}{metric,dinbold,OT1}}%
    \file{0m6t1.mtx}{\from{tudscr-metrics.dtx}{metric,dinbold,T1}}%
    \file{0m6ts1.mtx}{\from{tudscr-metrics.dtx}{metric,dinbold,TS1}}%
    \file{0m6oml.mtx}{\from{tudscr-metrics.dtx}{metric,dinbold,OML}}%
    \file{0m6oms.mtx}{\from{tudscr-metrics.dtx}{metric,dinbold,OMS}}%
    \file{0m6symb.mtx}{\from{tudscr-metrics.dtx}{metric,dinbold,symbols}}%
    \file{0m6bkern.mtx}{\from{tudscr-metrics.dtx}{metric,dinbold,kerning,up}}%
    \file{0m6bokern.mtx}{\from{tudscr-metrics.dtx}{metric,dinbold,kerning,sl}}%
  }
\else
  \let\endbatchfile\relax
\fi
\endbatchfile
%</ins>
%<*dtx>
% \fi
%
\ifx\ProvidesFile\undefined\def\ProvidesFile#1[#2]{}\fi
\ProvidesFile{tudscr-metrics.dtx}[2022/07/28 v2.06 TUD-Script\space%
  (font metrics)%
]
%
% \iffalse
\documentclass[english,ngerman,xindy]{tudscrdoc}
\iftutex
  \usepackage{fontspec}
\else
  \usepackage[T1]{fontenc}
  \usepackage[ngerman=ngerman-x-latest]{hyphsubst}
\fi
\usepackage{babel}
\usepackage{tudscrfonts}
\usepackage[babel]{microtype}

\CodelineIndex
\RecordChanges
\GetFileInfo{tudscr-metrics.dtx}
\title{\file{\filename}}
\author{Falk Hanisch\qquad\expandafter\mailto\expandafter{\tudscrmail}}
\date{\fileversion\nobreakspace(\filedate)}

\begin{document}
  \maketitle
  \tableofcontents
  \DocInput{\filename}
\end{document}
%</dtx>
% \fi
%
% \selectlanguage{ngerman}
%
% \changes{v2.04}{2015/04/22}{Fix für \cs{substitutesilent}}^^A
%
%
%
% \section{Routinen und Metriken für die Schriftinstallation}
%
% Hier werden die Routinen und Metrikdateien für die Schriftinstallation 
% mithilfe des Paketes \pkg{fontinst} generiert.
%
% \iffalse
%<*install>
% \fi
%
% \subsection{Installationsroutinen}
%
% Mit der Datei \file{installfonts.tex} werden die Schriften des \CDs
% (\Univers und \DIN) mit dem Paket \pkg{fontinst} installiert.
%    \begin{macrocode}
\input fontinst.sty
\needsfontinstversion{1.933}
%    \end{macrocode}
% Zuerst werden die Schriften aus den Paketen \pkg{cmbright} und \pkg{iwona},
% welche durch das überlagerte Installationsskript bereits mit \file{tftopl} 
% in das \file{.pl}-Format gewandelt wurden, in das \file{.mtx}-Format von 
% \pkg{fontinst} gewandelt.
%    \begin{macrocode}
\typeout{installfonts.tex: generalpltomtx}
\generalpltomtx{cmbr10}{cmbr10}{pl}{ot1}
\generalpltomtx{cmbrsl10}{cmbrsl10}{pl}{ot1}
\generalpltomtx{cmbrbx10}{cmbrbx10}{pl}{ot1}
\generalpltomtx{tbmr10}{tbmr10}{pl}{ts1}
\generalpltomtx{tbmo10}{tbmo10}{pl}{ts1}
\generalpltomtx{tbsr10}{tbsr10}{pl}{ts1}
\generalpltomtx{tbso10}{tbso10}{pl}{ts1}
\generalpltomtx{tbbx10}{tbbx10}{pl}{ts1}
\generalpltomtx{cmbrmi10}{cmbrmi10}{pl}{oml}
\generalpltomtx{cmbrmb10}{cmbrmb10}{pl}{oml}
\generalpltomtx{cmbrsy10}{cmbrsy10}{pl}{oms}
\generalpltomtx{sy-iwonamz}{sy-iwonamz}{pl}{oms}
\generalpltomtx{sy-iwonahz}{sy-iwonahz}{pl}{oms}
\generalpltomtx{rm-iwonach}{rm-iwonach}{pl}{ot1}
\generalpltomtx{rm-iwonachi}{rm-iwonachi}{pl}{ot1}
\generalpltomtx{ts1-iwonach}{ts1-iwonach}{pl}{ts1}
\generalpltomtx{ts1-iwonachi}{ts1-iwonachi}{pl}{ts1}
\generalpltomtx{mi-iwonachi}{mi-iwonachi}{pl}{oml}
\generalpltomtx{sy-iwonachz}{sy-iwonachz}{pl}{oms}
%    \end{macrocode}
% Die Schriften aus dem \file{.afm}-Format werden in das \file{.mtx}-Format 
% transformiert. 
%    \begin{macrocode}
\typeout{installfonts.tex: transformfont lun}
\transformfont{lunl8r}{\reencodefont{8r}{\fromafm{lunl8a}}}
\transformfont{lunlo8r}{\reencodefont{8r}{\fromafm{lunlo8a}}}
\transformfont{lunr8r}{\reencodefont{8r}{\fromafm{lunr8a}}}
\transformfont{lunro8r}{\reencodefont{8r}{\fromafm{lunro8a}}}
\transformfont{lunb8r}{\reencodefont{8r}{\fromafm{lunb8a}}}
\transformfont{lunbo8r}{\reencodefont{8r}{\fromafm{lunbo8a}}}
\transformfont{lunc8r}{\reencodefont{8r}{\fromafm{lunc8a}}}
\transformfont{lunco8r}{\reencodefont{8r}{\fromafm{lunco8a}}}
%    \end{macrocode}
% Danach wird das soeben erzeugte \file{.pl}-Format verfeinert. Dabei werden
% mit der Metrikdatei \file{ligature.mtx} fehlende Ligaturen ergänzt sowie
% Kerning bei den Ziffern entfernt.
%    \begin{macrocode}
\typeout{installfonts.tex: installrawfont lun}
\installfonts
\installrawfont{lunl8r}{lunl8r,ligature,8r}{8r}{8r}{lun}{l}{n}{}
\installrawfont{lunlo8r}{lunlo8r,ligature,8r}{8r}{8r}{lun}{l}{sl}{}
\installrawfont{lunr8r}{lunr8r,ligature,8r}{8r}{8r}{lun}{m}{n}{}
\installrawfont{lunro8r}{lunro8r,ligature,8r}{8r}{8r}{lun}{m}{sl}{}
\installrawfont{lunb8r}{lunb8r,ligature,8r}{8r}{8r}{lun}{b}{n}{}
\installrawfont{lunbo8r}{lunbo8r,ligature,8r}{8r}{8r}{lun}{b}{sl}{}
\installrawfont{lunc8r}{lunc8r,ligature,8r}{8r}{8r}{lun}{eb}{n}{}
\installrawfont{lunco8r}{lunco8r,ligature,8r}{8r}{8r}{lun}{eb}{sl}{}
%    \end{macrocode}
% Danach werden die Schriften installiert. Dabei kommen unterschiedliche 
% Metriken zum Einsatz. Als erstes die Schriftfamilie der \Univers \file{lun}
% in den gebräuchlichsten Kodierungen.
%    \begin{macrocode}
\typeout{installfonts.tex: installfamily lun}
\installfamily{OT1}{lun}{\hyphenchar\font=45}
\installfamily{T1}{lun}{\hyphenchar\font=45}
\installfamily{TS1}{lun}{\hyphenchar\font=-1}
\installfamily{OML}{lun}{\skewchar\font=127}
\installfamily{OMS}{lun}{\skewchar\font=48}
%    \end{macrocode}
% Anschließend werden für die einzelnen Schriftschnitte die jeweiligen 
% virtuellen Schrfiten erzeugt. Dabei werden für die \file{OT1}-Kodierung die
% griechischen Lettern aus dem \pkg{cmbright}-Paket verwendet. Für die
% \file{TS1}-Kodierung werden fehlende Zeichen ebenfalls aus diesem Paket 
% ersetzt. Zusätzlich wird für jede der Kodierungen \file{OT1}, \file{T1} 
% und \file{TS1} eine spiezielle Metrik \file{lun\meta{Kodierung}.mtx} für 
% die \Univers genutzt.
% \begingroup
% \hfuzz=50pt
%    \begin{macrocode}
\typeout{installfonts.tex: installfont lun}
\installfont{lunl7t}{lunl8r,lunot1,cmbr10,newlatin}{OT1}{OT1}{lun}{l}{n}{}
\installfont{lunl8t}{lunl8r,lunt1,newlatin}{T1}{T1}{lun}{l}{n}{}
\installfont{lunl8c}{lunl8r,lunts1,tbmr10,textcomp}{TS1}{TS1}{lun}{l}{n}{}
\installfont{lunlo7t}{lunlo8r,lunot1,cmbrsl10,newlatin}{OT1}{OT1}{lun}{l}{sl}{}
\installfont{lunlo8t}{lunlo8r,lunt1,newlatin}{T1}{T1}{lun}{l}{sl}{}
\installfont{lunlo8c}{lunlo8r,lunts1,tbmo10,textcomp}{TS1}{TS1}{lun}{l}{sl}{}
\installfont{lunlc7t}{lunl8r,lunot1,cmbr10,newlatin}{OT1c}{OT1}{lun}{l}{sc}{}
\installfont{lunlc8t}{lunl8r,lunt1,newlatin}{T1c}{T1}{lun}{l}{sc}{}
\installfont{lunr7t}{lunr8r,lunot1,cmbr10,newlatin}{OT1}{OT1}{lun}{m}{n}{}
\installfont{lunr8t}{lunr8r,lunt1,newlatin}{T1}{T1}{lun}{m}{n}{}
\installfont{lunr8c}{lunr8r,lunts1,tbsr10,textcomp}{TS1}{TS1}{lun}{m}{n}{}
\installfont{lunro7t}{lunro8r,lunot1,cmbrsl10,newlatin}{OT1}{OT1}{lun}{m}{sl}{}
\installfont{lunro8t}{lunro8r,lunt1,newlatin}{T1}{T1}{lun}{m}{sl}{}
\installfont{lunro8c}{lunro8r,lunts1,tbso10,textcomp}{TS1}{TS1}{lun}{m}{sl}{}
\installfont{lunrc7t}{lunr8r,lunot1,cmbr10,newlatin}{OT1c}{OT1}{lun}{m}{sc}{}
\installfont{lunrc8t}{lunr8r,lunt1,newlatin}{T1c}{T1}{lun}{m}{sc}{}
\installfont{lunb7t}{lunb8r,lunot1,cmbrbx10,newlatin}{OT1}{OT1}{lun}{b}{n}{}
\installfont{lunb8t}{lunb8r,lunt1,newlatin}{T1}{T1}{lun}{b}{n}{}
\installfont{lunb8c}{lunb8r,lunts1,tbbx10,textcomp}{TS1}{TS1}{lun}{b}{n}{}
\installfont{lunbo7t}{lunbo8r,lunot1,cmbrmb10,cmbrsl10,newlatin}{OT1}{OT1}{lun}{b}{sl}{}
\installfont{lunbo8t}{lunbo8r,lunt1,newlatin}{T1}{T1}{lun}{b}{sl}{}
\installfont{lunbo8c}{lunbo8r,lunts1,tbbx10,textcomp}{TS1}{TS1}{lun}{b}{sl}{}
\installfont{lunbc7t}{lunb8r,lunot1,cmbrbx10,newlatin}{OT1c}{OT1}{lun}{b}{sc}{}
\installfont{lunbc8t}{lunb8r,lunt1,newlatin}{T1c}{T1}{lun}{b}{sc}{}
\installfont{lunc7t}{lunc8r,lunot1,cmbrbx10,newlatin}{OT1}{OT1}{lun}{eb}{n}{}
\installfont{lunc8t}{lunc8r,lunt1,newlatin}{T1}{T1}{lun}{eb}{n}{}
\installfont{lunc8c}{lunc8r,lunts1,tbbx10,textcomp}{TS1}{TS1}{lun}{eb}{n}{}
\installfont{lunco7t}{lunco8r,lunot1,cmbrmb10,cmbrsl10,newlatin}{OT1}{OT1}{lun}{eb}{sl}{}
\installfont{lunco8t}{lunco8r,lunt1,newlatin}{T1}{T1}{lun}{eb}{sl}{}
\installfont{lunco8c}{lunco8r,lunts1,tbbx10,textcomp}{TS1}{TS1}{lun}{eb}{sl}{}
\installfont{luncc7t}{lunc8r,lunot1,cmbrbx10,newlatin}{OT1c}{OT1}{lun}{eb}{sc}{}
\installfont{luncc8t}{lunc8r,lunt1,newlatin}{T1c}{T1}{lun}{eb}{sc}{}
\substitutesilent{l}{l}
%    \end{macrocode}
% \endgroup
% Für die Kodierung der Mathematikschrift werden die Kursiven als Basisschrift 
% genutzt. Allerdings werden aufrechte Ziffern benötigt. Die griechischen 
% Lettern kommen aus dem Paket \pkg{cmbright}.
% \begingroup
% \hfuzz=70pt
%    \begin{macrocode}
\installfont{lunl7m}{lunlo8r,lunoml,lunl8r,lunsymb,cmbrmi10,mathit}{OML}{OML}{lun}{l}{sl}{}
\installfont{lunr7m}{lunro8r,lunoml,lunr8r,lunsymb,cmbrmi10,mathit}{OML}{OML}{lun}{m}{sl}{}
\installfont{lunb7m}{lunbo8r,lunoml,lunb8r,lunsymb,cmbrmb10,mathit}{OML}{OML}{lun}{b}{sl}{}
\installfont{lunc7m}{lunco8r,lunoml,lunc8r,lunsymb,cmbrmb10,mathit}{OML}{OML}{lun}{eb}{sl}{}
\substitutesilent{l}{l}
%    \end{macrocode}
% \endgroup
% Unbekannte Symbole in der \Univers werden aus dem Paket \pkg{iwona} ergänzt.
% \begingroup
% \hfuzz=25pt
%    \begin{macrocode}
\installfont{lunl7y}{lunl8r,lunoms,lunsymb,sy-iwonamz,mathsy}{OMS}{OMS}{lun}{l}{n}{}
\installfont{lunr7y}{lunr8r,lunoms,lunsymb,sy-iwonamz,mathsy}{OMS}{OMS}{lun}{m}{n}{}
\installfont{lunb7y}{lunb8r,lunoms,lunsymb,sy-iwonahz,mathsy}{OMS}{OMS}{lun}{b}{n}{}
\installfont{lunc7y}{lunc8r,lunoms,lunsymb,sy-iwonahz,mathsy}{OMS}{OMS}{lun}{eb}{n}{}
\substitutesilent{l}{l}
\endinstallfonts
%    \end{macrocode}
% \endgroup
% Zum Schluss werden aus dem Paket \pkg{lmodern} die Schreibmaschinenschriften 
% entnommen.
%    \begin{macrocode}
\typeout{installfonts.tex: installfamily lunt}
\installfonts
\installfamily{OT1}{lunt}{\hyphenchar\font=-1}
\installfamily{T1}{lunt}{\hyphenchar\font=-1}
\installfamily{TS1}{lunt}{\hyphenchar\font=-1}
\installfontas{rm-lmtl10}{OT1}{lunt}{l}{n}{}
\installfontas{rm-lmtlo10}{OT1}{lunt}{l}{sl}{}
\installfontas{rm-lmtt10}{OT1}{lunt}{m}{n}{}
\installfontas{rm-lmtto10}{OT1}{lunt}{m}{sl}{}
\installfontas{rm-lmtk10}{OT1}{lunt}{b}{n}{}
\installfontas{rm-lmtko10}{OT1}{lunt}{b}{sl}{}
\installfontas{ec-lmtl10}{T1}{lunt}{l}{n}{}
\installfontas{ec-lmtlo10}{T1}{lunt}{l}{sl}{}
\installfontas{ec-lmtt10}{T1}{lunt}{m}{n}{}
\installfontas{ec-lmtto10}{T1}{lunt}{m}{sl}{}
\installfontas{ec-lmtk10}{T1}{lunt}{b}{n}{}
\installfontas{ec-lmtko10}{T1}{lunt}{b}{sl}{}
\installfontas{ts1-lmtl10}{TS1}{lunt}{l}{n}{}
\installfontas{ts1-lmtlo10}{TS1}{lunt}{l}{sl}{}
\installfontas{ts1-lmtt10}{TS1}{lunt}{m}{n}{}
\installfontas{ts1-lmtto10}{TS1}{lunt}{m}{sl}{}
\installfontas{ts1-lmtk10}{TS1}{lunt}{b}{n}{}
\installfontas{ts1-lmtko10}{TS1}{lunt}{b}{sl}{}
\substitutesilent{l}{l}
\substitutesilent{eb}{bx}
\substitutesilent{bx}{b}
\endinstallfonts
%    \end{macrocode}
% Nach der \Univers folgt \DIN. Auch werden die Schriften erst transformiert
% und anschließend verfeinert.
%    \begin{macrocode}
\typeout{installfonts.tex: transformfont 0m6}
\transformfont{0m6b8r}{\reencodefont{8r}{\fromafm{0m6b8a}}}
\transformfont{0m6bo8r}{\slantfont{167}{\frommtx{0m6b8r}}}
\typeout{installfonts.tex: installrawfont 0m6}
\installfonts
\installrawfont{0m6b8r}{0m6b8r,ligature,8r}{8r}{8r}{0m6}{b}{n}{}
\installrawfont{0m6bo8r}{0m6bo8r,ligature,8r}{8r}{8r}{0m6}{b}{sl}{}
%    \end{macrocode}
% Nun die eigentliche Schriftfamilie \file{0m6}.
%    \begin{macrocode}
\typeout{installfonts.tex: installfamily 0m6}
\installfamily{OT1}{0m6}{\hyphenchar\font=45}
\installfamily{T1}{0m6}{\hyphenchar\font=45}
\installfamily{TS1}{0m6}{\hyphenchar\font=-1}
\installfamily{OML}{0m6}{\skewchar\font=127}
\installfamily{OMS}{0m6}{\skewchar\font=48}
%    \end{macrocode}
% Ähnlich wie bei der \Univers wird auch bei der \DIN verfahren. Symbole und 
% mathematische Schriftzeichen (inklusive griechischer Lettern) werden für
% die \DIN aus dem Paket \pkg{iwona} ergänzt. Zusätzlich muss das Kerning bei
% einigen Letternkombinationen angepasst werden, da dieses bei der
% Ausgangsschrift fehlerhaft ist.
% \changes{v2.05}{2016/04/07}{\DIN mit \cs{substitutesilent} für alle gängigen 
%   Schriftschnitte}^^A
% \begingroup
% \hfuzz=80pt
%    \begin{macrocode}
\typeout{installfonts.tex: installfont 0m6}
\installfont{0m6b7t}{0m6bkern,0m6b8r,0m6ot1,rm-iwonach,newlatin}{OT1}{OT1}{0m6}{b}{n}{}
\installfont{0m6b8t}{0m6bkern,0m6b8r,0m6t1,newlatin}{T1}{T1}{0m6}{b}{n}{}
\installfont{0m6b8c}{0m6bkern,0m6b8r,0m6ts1,ts1-iwonach,textcomp}{TS1}{TS1}{0m6}{b}{n}{}
\installfont{0m6bo7t}{0m6bokern,0m6bo8r,0m6ot1,rm-iwonachi,newlatin}{OT1}{OT1}{0m6}{b}{sl}{}
\installfont{0m6bo8t}{0m6bokern,0m6bo8r,0m6t1,newlatin}{T1}{T1}{0m6}{b}{sl}{}
\installfont{0m6bo8c}{0m6bokern,0m6bo8r,0m6ts1,ts1-iwonachi,textcomp}{TS1}{TS1}{0m6}{b}{sl}{}
\installfont{0m6bc7t}{0m6bkern,0m6b8r,0m6ot1,rm-iwonach,newlatin}{OT1c}{OT1}{0m6}{b}{sc}{}
\installfont{0m6bc8t}{0m6bkern,0m6b8r,0m6t1,newlatin}{T1c}{T1}{0m6}{b}{sc}{}
\installfont{0m6b7m}{0m6bo8r,0m6oml,0m6b8r,0m6symb,mi-iwonachi,mathit}{OML}{OML}{0m6}{b}{sl}{}
\installfont{0m6b7y}{0m6b8r,0m6oms,0m6symb,sy-iwonachz,mathsy}{OMS}{OMS}{0m6}{b}{n}{}
\substitutesilent{l}{l}
\substitutesilent{m}{m}
\substitutesilent{l}{m}
\substitutesilent{m}{eb}
\substitutesilent{eb}{bx}
\substitutesilent{bx}{b}
\endinstallfonts
\bye
%    \end{macrocode}
% \endgroup
%
% \iffalse
%</install>
%<*map>
% \fi
%
% \subsection{Zuordnungsliste der Schriften (fontmap)}
%
% Da es beim automatischen Generieren einer map-Datei Probleme gab, wird die
% Erstellung dieser manuell durchgeführt.
%    \begin{macrocode}
\input finstmsc.sty
\resetstr{PSfontsuffix}{.pfb}
\adddriver{dvips}{tudscr.map}
\storemapdata{lunr8a}{\fromafm{lunr8a}{UniversCE-Medium}}{}
\storemapdata{lunr8r}{\frommtx{lunr8a}}{\reencodefont{8r}}
\makemapentry{lunr8r}
\storemapdata{lunro8a}{\fromafm{lunro8a}{UniversCE-Oblique}}{}
\storemapdata{lunro8r}{\frommtx{lunro8a}}{\reencodefont{8r}}
\makemapentry{lunro8r}
\storemapdata{lunb8a}{\fromafm{lunb8a}{UniversCE-Bold}}{}
\storemapdata{lunb8r}{\frommtx{lunb8a}}{\reencodefont{8r}}
\makemapentry{lunb8r}
\storemapdata{lunbo8a}{\fromafm{lunbo8a}{UniversCE-BoldOblique}}{}
\storemapdata{lunbo8r}{\frommtx{lunbo8a}}{\reencodefont{8r}}
\makemapentry{lunbo8r}
\storemapdata{lunc8a}{\fromafm{lunc8a}{UniversCE-Black}}{}
\storemapdata{lunc8r}{\frommtx{lunc8a}}{\reencodefont{8r}}
\makemapentry{lunc8r}
\storemapdata{lunco8a}{\fromafm{lunco8a}{UniversCE-BlackOblique}}{}
\storemapdata{lunco8r}{\frommtx{lunco8a}}{\reencodefont{8r}}
\makemapentry{lunco8r}
\storemapdata{lunl8a}{\fromafm{lunl8a}{UniversCE-Light}}{}
\storemapdata{lunl8r}{\frommtx{lunl8a}}{\reencodefont{8r}}
\makemapentry{lunl8r}
\storemapdata{lunlo8a}{\fromafm{lunlo8a}{UniversCE-LightOblique}}{}
\storemapdata{lunlo8r}{\frommtx{lunlo8a}}{\reencodefont{8r}}
\makemapentry{lunlo8r}
\storemapdata{0m6b8a}{\fromafm{0m6b8a}{DIN-Bold}}{}
\storemapdata{0m6b8r}{\frommtx{0m6b8a}}{\reencodefont{8r}}
\makemapentry{0m6b8r}
\storemapdata{0m6bo8r}{\frommtx{0m6b8r}}{\transformfont{1000}{167}}
\makemapentry{0m6bo8r}
\donedrivers
\bye
%    \end{macrocode}
%
% \iffalse
%</map>
%<*metric>
% \fi
%
% \subsection{Metriken zur Schriftinstallation}
%
% Um die Basisschriften für alle Kodierungen verwendbar zu machen, bedarf es 
% einiger Anpassungen. Beispielsweise werden einzelne Zeichen, die für eine 
% bestimmte Kodierung zwingend notwendig sind, von den Ausgangschriften 
% \emph{nicht} bereitgestellt. Diese werden durch möglichst ähnliche aus
% verschiedenen \LaTeX"=Schriftpaketen aufgefüllt.
%
%    \begin{macrocode}
\relax
\metrics
\needsfontinstversion{1.933}
%    \end{macrocode}
%
% \iffalse
%<*general>
% \fi
%
% \subsubsection{Allgemeine Metriken}
%
% Diese Metriken kommen bei der Generierung beider Schriftfamilien~-- also 
% sowohl \Univers als auch \DIN~-- zum Einsatz.
%
% \iffalse
%<*ligature>
% \fi
%
% \paragraph{Ligaturen}
%
% Damit werden fehlende Ligaturen ergänzt und das Kerning bei Ziffern entfernt.
%    \begin{macrocode}
\setglyph{ff}
  \glyph{f}{1000}
  \glyph{f}{1000}
\endsetglyph
\setglyph{ffi}
  \glyph{f}{1000}
  \glyph{fi}{1000}
\endsetglyph
\setglyph{ffl}
  \glyph{f}{1000}
  \glyph{fl}{1000}
\endsetglyph
\noleftrightkerning{hyphen}
\noleftrightkerning{zero}
\noleftrightkerning{one}
\noleftrightkerning{two}
\noleftrightkerning{three}
\noleftrightkerning{four}
\noleftrightkerning{five}
\noleftrightkerning{six}
\noleftrightkerning{seven}
\noleftrightkerning{eight}
\noleftrightkerning{nine}
%    \end{macrocode}
%
% \iffalse
%</ligature>
%<*greek>
% \fi
%
% \paragraph{Griechische Lettern}
%
% Damit werden bereits definierte griechische Lettern entfernt. Bei den 
% Ausgangsschriften sind teilweise griechische Lettern vorhanden~-- zumeist
% jedoch lediglich Delta ($\Delta$) und mu ($\mu$). Um ein einheitliches Bild
% bei allen griechischen Lettern zu erhalten und diese vollständig aus einer
% Ersatzschrift zu verwenden, ist diese Metrik notwendig.
%    \begin{macrocode}
\unsetglyph{Gamma}
\unsetglyph{Delta}
\unsetglyph{Theta}
\unsetglyph{Lambda}
\unsetglyph{Xi}
\unsetglyph{Pi}
\unsetglyph{Sigma}
\unsetglyph{Upsilon}
\unsetglyph{Phi}
\unsetglyph{Psi}
\unsetglyph{Omega}
\unsetglyph{mu}
%    \end{macrocode}
%
% \iffalse
%</greek>
%<*numbers>
% \fi
%
% \paragraph{Mediäval- oder Minuskelziffern}
%
% Hiermit werden Ziffern für den Fließtext nachgeahmt.
%    \begin{macrocode}
\setglyph{zerooldstyle}
  \glyph{zero}{1000}
\endsetglyph
\setglyph{oneoldstyle}
  \glyph{one}{1000}
\endsetglyph
\setglyph{twooldstyle}
  \glyph{two}{1000}
\endsetglyph
\setglyph{threeoldstyle}
  \moveup{-190}
  \glyph{three}{1000}
  \moveup{190}
\endsetglyph
\setglyph{fouroldstyle}
  \moveup{-190}
  \glyph{four}{1000}
  \moveup{190}
\endsetglyph
\setglyph{fiveoldstyle}
  \moveup{-190}
  \glyph{five}{1000}
  \moveup{190}
\endsetglyph
\setglyph{sixoldstyle}
  \glyph{six}{1000}
\endsetglyph
\setglyph{sevenoldstyle}
  \moveup{-190}
  \glyph{seven}{1000}
  \moveup{190}
\endsetglyph
\setglyph{eightoldstyle}
  \glyph{eight}{1000}
\endsetglyph
\setglyph{nineoldstyle}
  \moveup{-190}
  \glyph{nine}{1000}
  \moveup{190}
\endsetglyph
%    \end{macrocode}
%
% \iffalse
%</numbers>
%</general>
%<*univers|dinbold>
% \fi
%
% \subsubsection{Metriken für die Installation von \Univers und \DIN}
%
% Zunächst ein paar allgemeine Einstellungen für die Wortzwischenräume.
% \changes{v2.05}{2016/04/18}{Bugfix für Wortzwischenräume bei \DIN}^^A
%    \begin{macrocode}
%<*OT1|T1|TS1>
\ifisglyph{space}\then
  \setint{interword}{\width{space}}
\else\ifisglyph{i}\then
  \setint{interword}{\width{i}}
\else
  \setint{interword}{333}
\fi\fi
\setint{extraspace}{\scale{\int{interword}}{333}}
%</OT1|T1|TS1>
%<*OML|OMS>
\setint{interword}{0}
\setint{extraspace}{0}
%</OML|OMS>
%<*OT1|T1>
\setint{stretchword}{\scale{\int{interword}}{500}}
\setint{shrinkword}{\scale{\int{interword}}{333}}
\setint{quad}{\scale{\int{interword}}{3600}}
%</OT1|T1>
%<*TS1|OML|OMS>
\setint{stretchword}{0}
\setint{shrinkword}{0}
\setint{quad}{1000}
%</TS1|OML|OMS>
%    \end{macrocode}
%
% \iffalse
%<*univers>
% \fi
%
% \subsubsection{Metriken für die Installation von \Univers}
%
% Diese Metriken kommen bei der Installation der \Univers zum Einsatz.
%
% \iffalse
%<*OT1>
% \fi
%
% \paragraph{OT1-Kodierung}
%
% Neben den bekannten Metriken wird |lslashslash| ersetzt, um diese Glyphe aus
% \pkg{cmbright} zu verwenden.
%    \begin{macrocode}
\inputmtx{ligature}
\inputmtx{newlatin}
\inputmtx{unsetgreek}
\unsetglyph{lslashslash}
%    \end{macrocode}
%
% \iffalse
%</OT1>
%<*T1>
% \fi
%
% \paragraph{T1-Kodierung}
%
% Die Glyphe |perthousandzero| wird~-- leider sehr ungünstig~-- imitiert.
%    \begin{macrocode}
\setglyph{perthousandzero}
  \moveup{-4}
  \glyph{o}{711}
  \moveup{4}
\endsetglyph
\setkern{percent}{perthousandzero}{-81}
\setkern{perthousandzero}{perthousandzero}{-65}
\inputmtx{ligature}
%    \end{macrocode}
%
% \iffalse
%</T1>
%<*TS1>
% \fi
%
% \paragraph{TS1-Kodierung}
%
% Für die \file{TS1}-Kodierung wird enormer Aufwand getrieben. In dieser 
% befinden sich viele, letternähnliche Symbole, die im Folgenden nachgeahmt
% werden.
% \begingroup
% \hfuzz=60pt
%    \begin{macrocode}
\inputmtx{unsetgreek}
\inputmtx{setoldnum}
\setint{strokewidth}{\add{\height{minus}}{\depth{minus}}}
\setcommand\hstroke#1#2#3{
  \moveup{\scale{\height{#1}}{#2}}
  \movert{\scale{\width{#1}}{#3}}
  \movert{\div{\mul{\scale{\height{#1}}{#2}}{\int{italicslant}}}{1000}}
  \glyphrule{\mul{\int{strokewidth}}{7}}{\int{strokewidth}}
}
\setcommand\hdblstroke#1#2#3{
  \moveup{\scale{\height{#1}}{300}}
  \movert{\scale{\width{#1}}{#3}}
  \movert{\div{\mul{\scale{\height{#1}}{300}}{\int{italicslant}}}{1000}}
  \push
    \glyphrule{\scale{\width{#1}}{#2}}{\int{strokewidth}}
  \pop
  \moveup{\scale{\height{#1}}{300}}
  \movert{\div{\mul{\scale{\height{#1}}{300}}{\int{italicslant}}}{1000}}
  \glyphrule{\scale{\width{#1}}{#2}}{\int{strokewidth}}
}
\setcommand\vstroke#1{
  \resetint{scaledbar}{\scale{\height{#1}}{1200}}
  \moveup{\neg{\add{\scale{\height{#1}}{100}}{\depth{#1}}}}
  \movert{\add{\scale{\width{#1}}{500}}{\neg{\half{\scale{\width{bar}}{\int{scaledbar}}}}}}
  \movert{\half{\half{\scale{\width{#1}}{\int{italicslant}}}}}
  \moveup{\scale{\depth{bar}}{\int{scaledbar}}}
  \glyph{bar}{\int{scaledbar}}
}
\resetglyph{blank}
  \push
    \glyph{b}{1000}
  \pop
  \hstroke{b}{800}{-100}
  \samesize{b}
\endresetglyph
\resetglyph{hyphendbl}
  \push
    \glyph{hyphen}{1000}
  \pop
  \moveup{\mul{\add{\height{hyphen}}{\depth{hyphen}}}{3}}
  \movert{\div{\mul{\height{hyphen}}{\int{italicslant}}}{1000}}
  \glyph{hyphen}{1000}
\endresetglyph
\resetglyph{hyphendblchar}
  \glyph{hyphendbl}{1000}
\endresetglyph
\resetglyph{bardbl}
  \glyph{bar}{1000}
  \glyph{bar}{1000}
\endresetglyph
\resetglyph{centigrade}
  \glyph{degree}{1000}
  \glyph{C}{1000}
  \resetitalic{\italic{C}}
\endresetglyph
\resetglyph{dollaroldstyle}
  \push
    \glyph{S}{1000}
  \pop
  \push
    \movert{\neg{\int{strokewidth}}}
    \vstroke{S}
  \pop
  \movert{\int{strokewidth}}
  \vstroke{S}
  \samesize{S}
\endresetglyph
\resetglyph{centoldstyle}
  \glyph{cent}{1000}
\endresetglyph
\resetglyph{colonmonetary}
  \push
    \glyph{C}{1000}
  \pop
  \vstroke{C}
  \samesize{C}
\endresetglyph
\resetglyph{won}
  \push
    \glyph{W}{1000}
  \pop
  \hdblstroke{W}{1100}{-50}
  \samesize{W}
\endresetglyph
\resetglyph{naira}
  \push
    \glyph{N}{1000}
  \pop
  \hdblstroke{N}{1200}{-100}
  \samesize{N}
\endresetglyph
\resetglyph{guarani}
  \push
    \glyph{G}{1000}
  \pop
  \vstroke{G}
  \samesize{G}
\endresetglyph
\resetglyph{peso}
  \push
    \glyph{P}{1000}
  \pop
  \hstroke{P}{200}{-100}
  \samesize{P}
\endresetglyph
\resetglyph{recipe}
  \push
    \glyph{R}{1000}
  \pop
  \hstroke{R}{200}{500}
  \samesize{R}
\endresetglyph
\resetglyph{dong}
  \push
    \glyph{d}{1000}
  \pop
  \push
    \hstroke{d}{800}{450}
  \pop
  \samesize{d}
  \movert{\neg{\width{d}}}
  \moveup{\neg{\scale{\height{d}}{150}}}
  \push
    \glyphrule{\width{d}}{\half{\int{strokewidth}}}
  \pop
  \moveup{\scale{\height{d}}{150}}
\endresetglyph
\resetglyph{pertenthousand}
  \glyph{perthousand}{1000}
  \moveup{\neg{\scale{\height{perthousand}}{610}}}
  \movert{\neg{\scale{\width{ordmasculine}}{\int{italicslant}}}}
  \glyph{ordmasculine}{1000}
\endresetglyph
\resetglyph{baht}
  \push
    \glyph{B}{1000}
  \pop
  \vstroke{B}
  \samesize{B}
\endresetglyph
\resetglyph{cent}
  \push
    \glyph{c}{1000}
  \pop
  \vstroke{c}
  \samesize{c}
\endresetglyph
\resetglyph{euro}
\setglyph{euro}
  \movert{50}
  \push
    \moveup{\half{\sub{\height{C}}{\depth{C}}}}
    \movert{\scale
      {\half{\sub{\height{C}}{\depth{C}}}}
      {\int{italicslant}}
    }
    \push
      \moveup{\scale{\int{underlinethickness}}{1500}}
      \glyphrule{\scale{\width{C}}{750}}{\int{underlinethickness}}
    \pop
    \moveup{\scale{\int{underlinethickness}}{-1500}}
    \glyphrule{\scale{\width{C}}{700}}{\int{underlinethickness}}
  \pop
  \movert{50}
  \glyph{C}{1000}
  \resetitalic{\italic{C}}
\endresetglyph
%    \end{macrocode}
% \endgroup
%
% \iffalse
%</TS1>
%<*OML>
% \fi
%
% \paragraph{OML-Kodierung}
%
% Für das Setzen der Akzente im Mathematikmodus wird das Kerning definiert. 
% Dazu wurde bei der Initialisierung der beiden Schriftfamilien \file{lun} und 
% \file{0m6} die Glyphe |tie| mit \cs{skewchar}\cs{font}|=127| gesetzt.
%    \begin{macrocode}
\inputmtx{unsetgreek}
\inputmtx{unsetnum}
\unsetglyph{less}
\unsetglyph{greater}
\setkern{A}{tie}{140}
\setkern{B}{tie}{110}
\setkern{C}{tie}{120}
\setkern{D}{tie}{100}
\setkern{E}{tie}{120}
\setkern{F}{tie}{120}
\setkern{G}{tie}{120}
\setkern{H}{tie}{120}
\setkern{I}{tie}{120}
\setkern{J}{tie}{240}
\setkern{K}{tie}{120}
\setkern{L}{tie}{50}
\setkern{M}{tie}{120}
\setkern{N}{tie}{120}
\setkern{O}{tie}{120}
\setkern{P}{tie}{100}
\setkern{Q}{tie}{120}
\setkern{R}{tie}{100}
\setkern{S}{tie}{100}
\setkern{T}{tie}{90}
\setkern{U}{tie}{120}
\setkern{V}{tie}{70}
\setkern{W}{tie}{80}
\setkern{X}{tie}{70}
\setkern{Y}{tie}{70}
\setkern{Z}{tie}{90}
\setkern{a}{tie}{100}
\setkern{b}{tie}{110}
\setkern{c}{tie}{80}
\setkern{d}{tie}{90}
\setkern{e}{tie}{80}
\setkern{f}{tie}{110}
\setkern{g}{tie}{70}
\setkern{h}{tie}{110}
\setkern{i}{tie}{110}
\setkern{j}{tie}{110}
\setkern{k}{tie}{110}
\setkern{l}{tie}{110}
\setkern{m}{tie}{110}
\setkern{n}{tie}{100}
\setkern{o}{tie}{90}
\setkern{p}{tie}{90}
\setkern{q}{tie}{80}
\setkern{r}{tie}{90}
\setkern{s}{tie}{80}
\setkern{t}{tie}{80}
\setkern{u}{tie}{90}
\setkern{v}{tie}{50}
\setkern{w}{tie}{50}
\setkern{x}{tie}{60}
\setkern{y}{tie}{50}
\setkern{z}{tie}{80}
%    \end{macrocode}
%
% \iffalse
%</OML>
%<*OMS>
% \fi
%
% \paragraph{OMS-Kodierung}
%
% Auch für die Symbole werden einige Glyphen entfernt und mit denen aus dem 
% Paket \pkg{eulervm} ersetzt, andere werden imitiert.
%    \begin{macrocode}
\inputmtx{unsetalf}
\resetglyph{bardbl}
  \glyph{bar}{1000}
  \glyph{bar}{1000}
\endresetglyph
\resetglyph{asteriskmath}
  \moveup{\neg{\sub
    {\scale{\sub{\height{asterisk}}{\depth{asterisk}}}{500}}
    {\scale{\sub{\height{parenleft}}{\depth{parenleft}}}{500}}
  }}
  \glyph{asterisk}{1000}
  \resetdepth{0}
\endresetglyph
\setglyph{emptysetstress}
  \push
    \movert{\div{\sub{\width{zero}}{\width{slash}}}{2}}
    \moveup{\div{\sub{
      \add{\height{zero}}{\depth{slash}}
    }{
      \add{\height{slash}}{\depth{zero}}
    }}{2}}
    \glyph{slash}{1000}
  \pop
  \glyph{zero}{1000}
\endsetglyph
%    \end{macrocode}
%
% \iffalse
%</OMS>
%<*symbols>
% \fi
%
%
% \paragraph{Symbole}
%
% Einige Symbole werden abschließend angepasst.
%    \begin{macrocode}
\inputmtx{unsetgreek}
\inputmtx{setoldnum}
\resetglyph{lessequal}
  \push
    \moveup{\scale{\add{\height{minus}}{\depth{minus}}}{1200}}
    \glyph{less}{1000}
  \pop
  \moveup{\neg{\half{\add{\height{less}}{\depth{less}}}}}
  \moveup{\neg{\scale{\add{\height{minus}}{\depth{minus}}}{1200}}}
  \glyph{minus}{1000}
\endresetglyph
\resetglyph{greaterequal}
  \push
    \moveup{\scale{\add{\height{minus}}{\depth{minus}}}{1200}}
    \glyph{greater}{1000}
  \pop
  \moveup{\neg{\half{\add{\height{greater}}{\depth{greater}}}}}
  \moveup{\neg{\scale{\add{\height{minus}}{\depth{minus}}}{1200}}}
  \glyph{minus}{1000}
\endresetglyph
\resetglyph{plusminus}
  \push
    \moveup{\scale{\add{\height{minus}}{\depth{minus}}}{1200}}
    \glyph{plus}{1000}
  \pop
  \moveup{\neg{\half{\add{\height{plus}}{\depth{plus}}}}}
  \moveup{\neg{\half{\add{\height{minus}}{\depth{minus}}}}}
  \glyph{minus}{1000}
\endresetglyph
\resetglyph{bullet}
  \moveup{\neg{\scale{\add{\height{bullet}}{\depth{bullet}}}{200}}}
  \glyph{bullet}{1000}
\endresetglyph
%    \end{macrocode}
%
%
% \iffalse
%</symbols>
%</univers>
%<*dinbold>
% \fi
%
% \subsubsection{Metriken für die Installation von \DIN}
%
% Diese Metriken kommen bei der Installation der \DIN zum Einsatz.
%
% \iffalse
%<*OT1>
% \fi
%
% \paragraph{OT1-Kodierung}
%
% Neben den bekannten Metriken werden die Glyphen |lslashslash| sowie |at| mit 
% denen aus \pkg{iwona} ersetzt.
%    \begin{macrocode}
\inputmtx{ligature}
\inputmtx{newlatin}
\inputmtx{unsetgreek}
\unsetglyph{at}
\unsetglyph{lslashslash}
%    \end{macrocode}
%
% \iffalse
%</OT1>
%<*T1>
% \fi
%
% \paragraph{T1-Kodierung}
%
% Die Glyphe |perthousandzero| wird (schlecht) nachgeahmt.
%    \begin{macrocode}
\setglyph{perthousandzero}
  \moveup{-1}
  \glyph{o}{702}
  \moveup{1}
\endsetglyph
\setkern{percent}{perthousandzero}{-45}
\setkern{perthousandzero}{perthousandzero}{-35}
\inputmtx{ligature}
%    \end{macrocode}
%
% \iffalse
%</T1>
%<*TS1>
% \fi
%
% \paragraph{TS1-Kodierung}
%
% Ähnlich wie bei \Univers wird auch bei \DIN in der \file{TS1}-Kodierung eine
% Menge an letternähnliche Glyphen imitiert.
% \begingroup
% \hfuzz=60pt
%    \begin{macrocode}
\inputmtx{unsetgreek}
\inputmtx{setoldnum}
\setint{strokewidth}{\scale{\add{\height{minus}}{\depth{minus}}}{750}}
\setcommand\hstroke#1#2#3{
  \moveup{\scale{\height{#1}}{#2}}
  \movert{\scale{\width{#1}}{#3}}
  \movert{\div{\mul{\scale{\height{#1}}{#2}}{\int{italicslant}}}{1000}}
  \glyphrule{\mul{\int{strokewidth}}{5}}{\int{strokewidth}}
}
\setcommand\hdblstroke#1#2#3{
  \moveup{\scale{\height{#1}}{300}}
  \movert{\scale{\width{#1}}{#3}}
  \movert{\div{\mul{\scale{\height{#1}}{300}}{\int{italicslant}}}{1000}}
  \push
    \glyphrule{\scale{\width{#1}}{#2}}{\int{strokewidth}}
  \pop
  \moveup{\scale{\height{#1}}{300}}
  \movert{\div{\mul{\scale{\height{#1}}{300}}{\int{italicslant}}}{1000}}
  \glyphrule{\scale{\width{#1}}{#2}}{\int{strokewidth}}
}
\setcommand\vstroke#1{
  \resetint{scaledbar}{\scale{\add{\height{#1}}{\depth{#1}}}{1400}}
  \movert{\add
    {\scale{\width{#1}}{500}}
    {\neg{\half{\scale{\width{bar}}{\int{scaledbar}}}}}
  }
  \glyph{bar}{\int{scaledbar}}
}
\resetglyph{blank}
  \push
    \glyph{b}{1000}
  \pop
  \hstroke{b}{800}{-100}
  \samesize{b}
\endresetglyph
\resetglyph{hyphendbl}
  \push
    \glyph{hyphen}{1000}
  \pop
  \moveup{\mul{\add{\height{hyphen}}{\depth{hyphen}}}{2}}
  \movert{\div{\mul{\height{hyphen}}{\int{italicslant}}}{1000}}
  \glyph{hyphen}{1000}
\endresetglyph
\resetglyph{hyphendblchar}
  \glyph{hyphendbl}{1000}
\endresetglyph
\setglyph{openbracketleft}
  \glyph{bracketleft}{1000}
  \movert{\neg{\scale{\width{bracketleft}}{450}}}
  \glyph{bracketleft}{1000}
\endsetglyph
\setglyph{openbracketright}
  \glyph{bracketright}{1000}
  \movert{\neg{\scale{\width{bracketright}}{450}}}
  \glyph{bracketright}{1000}
\endsetglyph
\resetglyph{bardbl}
  \glyph{bar}{1000}
  \movert{\neg{\scale{\width{bar}}{333}}}
  \glyph{bar}{1000}
\endresetglyph
\resetglyph{centigrade}
  \glyph{degree}{1000}
  \glyph{C}{1000}
  \resetitalic{\italic{C}}
\endresetglyph
\resetglyph{dollaroldstyle}
  \push
    \glyph{S}{1000}
  \pop
  \push
    \movert{\neg{\int{strokewidth}}}
    \vstroke{S}
  \pop
  \movert{\int{strokewidth}}
  \vstroke{S}
  \samesize{S}
\endresetglyph
\resetglyph{centoldstyle}
  \push
    \glyph{c}{1000}
  \pop
  \movert{\scale{\width{c}}{300}}
  \moveup{\neg{\scale{\height{fraction}}{50}}}
  \glyph{fraction}{850}
  \samesize{c}
\endresetglyph
\resetglyph{colonmonetary}
  \push
    \glyph{C}{1000}
  \pop
  \vstroke{C}
  \samesize{C}
\endresetglyph
\resetglyph{won}
  \push
    \glyph{W}{1000}
  \pop
  \hdblstroke{W}{1100}{-50}
  \samesize{W}
\endresetglyph
\resetglyph{naira}
  \push
    \glyph{N}{1000}
  \pop
  \hdblstroke{N}{1200}{-100}
  \samesize{N}
\endresetglyph
\resetglyph{guarani}
  \push
    \glyph{G}{1000}
  \pop
  \vstroke{G}
  \samesize{G}
\endresetglyph
\resetglyph{peso}
  \push
    \glyph{P}{1000}
  \pop
  \hstroke{P}{200}{-100}
  \samesize{P}
\endresetglyph
\resetglyph{recipe}
  \push
    \glyph{R}{1000}
  \pop
  \hstroke{R}{200}{400}
  \samesize{R}
\endresetglyph
\resetglyph{dong}
  \push
    \glyph{d}{1000}
  \pop
  \push
    \hstroke{d}{800}{300}
  \pop
  \samesize{d}
  \movert{\neg{\width{d}}}
  \moveup{\neg{\scale{\height{d}}{150}}}
  \push
    \glyphrule{\width{d}}{\half{\int{strokewidth}}}
  \pop
  \moveup{\scale{\height{d}}{150}}
\endresetglyph
\resetglyph{pertenthousand}
  \glyph{perthousand}{1000}
  \moveup{\neg{\scale{\height{perthousand}}{370}}}
  \movert{\neg{\scale{\width{perthousand}}{40}}}
  \movert{\neg{\scale{\width{ordmasculine}}{\half{\int{italicslant}}}}}
  \glyph{ordmasculine}{880}
\endsetglyph
\resetglyph{baht}
  \push
    \glyph{B}{1000}
  \pop
  \vstroke{B}
  \samesize{B}
\endresetglyph
\resetglyph{euro}
  \glyph{Euro}{1000}
\endresetglyph
%    \end{macrocode}
% \endgroup
%
% \iffalse
%</TS1>
%<*OML>
% \fi
%
% \paragraph{OML-Kodierung}
%
% Das Kerning der mathematischen Akzente für \DIN.
%    \begin{macrocode}
\inputmtx{unsetgreek}
\inputmtx{unsetnum}
\unsetglyph{less}
\unsetglyph{greater}
\setkern{A}{tie}{80}
\setkern{B}{tie}{70}
\setkern{C}{tie}{100}
\setkern{D}{tie}{50}
\setkern{E}{tie}{90}
\setkern{F}{tie}{90}
\setkern{G}{tie}{80}
\setkern{H}{tie}{70}
\setkern{I}{tie}{80}
\setkern{J}{tie}{180}
\setkern{K}{tie}{80}
\setkern{L}{tie}{60}
\setkern{M}{tie}{60}
\setkern{N}{tie}{60}
\setkern{O}{tie}{70}
\setkern{P}{tie}{60}
\setkern{Q}{tie}{70}
\setkern{R}{tie}{50}
\setkern{S}{tie}{70}
\setkern{T}{tie}{70}
\setkern{U}{tie}{70}
\setkern{V}{tie}{70}
\setkern{W}{tie}{80}
\setkern{X}{tie}{70}
\setkern{Y}{tie}{70}
\setkern{Z}{tie}{80}
\setkern{a}{tie}{50}
\setkern{b}{tie}{60}
\setkern{c}{tie}{60}
\setkern{d}{tie}{70}
\setkern{e}{tie}{60}
\setkern{f}{tie}{80}
\setkern{g}{tie}{50}
\setkern{h}{tie}{60}
\setkern{i}{tie}{80}
\setkern{j}{tie}{80}
\setkern{k}{tie}{80}
\setkern{l}{tie}{60}
\setkern{m}{tie}{80}
\setkern{n}{tie}{60}
\setkern{o}{tie}{50}
\setkern{p}{tie}{60}
\setkern{q}{tie}{50}
\setkern{r}{tie}{80}
\setkern{s}{tie}{60}
\setkern{t}{tie}{60}
\setkern{u}{tie}{50}
\setkern{v}{tie}{50}
\setkern{w}{tie}{50}
\setkern{x}{tie}{60}
\setkern{y}{tie}{50}
\setkern{z}{tie}{70}
\setkern{Gamma}{tie}{90}
\setkern{Delta}{tie}{100}
\setkern{Theta}{tie}{90}
\setkern{Lambda}{tie}{100}
\setkern{Xi}{tie}{90}
\setkern{Pi}{tie}{90}
\setkern{Sigma}{tie}{80}
\setkern{Upsilon}{tie}{60}
\setkern{Phi}{tie}{110}
\setkern{Psi}{tie}{90}
\setkern{Omega}{tie}{100}
\setkern{alpha}{tie}{50}
\setkern{beta}{tie}{80}
\setkern{gamma}{tie}{60}
\setkern{delta}{tie}{70}
\setkern{epsilon}{tie}{50}
\setkern{zeta}{tie}{90}
\setkern{eta}{tie}{70}
\setkern{theta}{tie}{90}
\setkern{iota}{tie}{60}
\setkern{kappa}{tie}{60}
\setkern{lambda}{tie}{50}
\setkern{mu}{tie}{60}
\setkern{nu}{tie}{50}
\setkern{xi}{tie}{80}
\setkern{pi}{tie}{60}
\setkern{rho}{tie}{60}
\setkern{sigma}{tie}{50}
\setkern{tau}{tie}{50}
\setkern{upsilon}{tie}{50}
\setkern{phi}{tie}{110}
\setkern{chi}{tie}{50}
\setkern{psi}{tie}{50}
\setkern{omega}{tie}{50}
\setkern{epsilon1}{tie}{50}
\setkern{theta1}{tie}{50}
\setkern{pi1}{tie}{60}
\setkern{rho1}{tie}{50}
\setkern{sigma1}{tie}{60}
\setkern{phi1}{tie}{60}
%    \end{macrocode}
%
% \iffalse
%</OML>
%<*OMS>
% \fi
%
% \paragraph{OMS-Kodierung}
%
% Bei den Symbolen werden einige Glyphen nachgeahmt, andere mit denen aus dem 
% Paket \pkg{iwona} ersetzt.
%    \begin{macrocode}
\inputmtx{unsetalf}
\resetglyph{bardbl}
  \glyph{bar}{1000}
  \glyph{bar}{1000}
\endresetglyph
\resetglyph{asteriskmath}
  \moveup{\neg{\sub
    {\scale{\sub{\height{asterisk}}{\depth{asterisk}}}{500}}
    {\scale{\sub{\height{parenleft}}{\depth{parenleft}}}{500}}
  }}
  \glyph{asterisk}{1000}
  \resetdepth{0}
\endresetglyph
\setglyph{emptysetstress}
  \push
    \movert{\div{\sub{\width{zero}}{\width{slash}}}{2}}
    \moveup{\div{\sub{
      \add{\height{zero}}{\depth{slash}}
    }{
      \add{\height{slash}}{\depth{zero}}
    }}{2}}
    \glyph{slash}{1000}
  \pop
  \glyph{zero}{1000}
\endsetglyph
%    \end{macrocode}
%
% \iffalse
%</OMS>
%<*symbols>
% \fi
%
%
% \paragraph{Symbole}
%
% Einige Symbole werden abschließend angepasst.
%    \begin{macrocode}
\inputmtx{unsetgreek}
\inputmtx{setoldnum}
\resetglyph{lessequal}
  \push
    \moveup{\scale{\add{\height{minus}}{\depth{minus}}}{600}}
    \glyph{less}{1000}
  \pop
  \moveup{\neg{\half{\add{\height{less}}{\depth{less}}}}}
  \moveup{\neg{\scale{\add{\height{minus}}{\depth{minus}}}{600}}}
  \glyph{minus}{1000}
\endresetglyph
\resetglyph{greaterequal}
  \push
    \moveup{\scale{\add{\height{minus}}{\depth{minus}}}{600}}
    \glyph{greater}{1000}
  \pop
  \moveup{\neg{\half{\add{\height{greater}}{\depth{greater}}}}}
  \moveup{\neg{\scale{\add{\height{minus}}{\depth{minus}}}{600}}}
  \glyph{minus}{1000}
\endresetglyph
\resetglyph{bullet}
  \moveup{\neg{\scale{\add{\height{bullet}}{\depth{bullet}}}{200}}}
  \glyph{bullet}{1000}
\endresetglyph
%    \end{macrocode}
%
% \iffalse
%</symbols>
%<*kerning>
% \fi
%
% \paragraph{Kerning}
%
% Für \DIN muss das Kerning bei einigen Letternkombinationen angepasst 
% werden, da dieses in der Ausgangsschrift fehlerhaft ist.
%
%    \begin{macrocode}
%<*up>
\setscaledrawglyph{C}{0m6b8r}{10pt}{1000}{67}{632}{718}{6}{0}
%</up>
%<*sl>
\setscaledrawglyph{C}{0m6bo8r}{10pt}{1000}{67}{632}{718}{6}{120}
%</sl>
\setkern{A}{V}{-85}
\setkern{A}{W}{-75}
\setkern{V}{A}{-85}
\setkern{W}{A}{-65}
\setkern{A}{A}{30}
\setkern{A}{J}{30}
\setkern{A}{S}{20}
\setkern{A}{X}{30}
\setkern{A}{Z}{20}
\setkern{C}{C}{10}
\setkern{C}{G}{10}
\setkern{C}{O}{10}
\setkern{C}{Q}{10}
\setkern{K}{A}{30}
\setkern{L}{A}{30}
\setkern{X}{A}{30}
\setkern{Y}{J}{-75}
\setkern{Y}{V}{20}
\setkern{Y}{W}{20}
\setkern{Y}{X}{20}
\setkern{Y}{Y}{20}
%    \end{macrocode}
%
% \iffalse
%</kerning>
%</dinbold>
%</univers|dinbold>
% \fi
%
%    \begin{macrocode}
\endmetrics
%    \end{macrocode}
%
% \iffalse
%</metric>
% \fi
%
% \PrintBackMatter
%
\endinput

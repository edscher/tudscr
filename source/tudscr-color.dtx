% \CheckSum{216}
% \iffalse meta-comment
%
%  TUD-Script -- Corporate Design of Technische Universität Dresden
% ----------------------------------------------------------------------------
%
%  Copyright (C) Falk Hanisch <hanisch.latex@outlook.com>, 2012-2020
%
% ----------------------------------------------------------------------------
%
%  This work may be distributed and/or modified under the conditions of the
%  LaTeX Project Public License, version 1.3c of the license. The latest
%  version of this license is in http://www.latex-project.org/lppl.txt and
%  version 1.3c or later is part of all distributions of LaTeX 2005/12/01
%  or later and of this work. This work has the LPPL maintenance status
%  "author-maintained". The current maintainer and author of this work
%  is Falk Hanisch.
%
% ----------------------------------------------------------------------------
%
%  Dieses Werk darf nach den Bedingungen der LaTeX Project Public Lizenz
%  in der Version 1.3c, verteilt und/oder verändert werden. Die aktuelle
%  Version dieser Lizenz ist http://www.latex-project.org/lppl.txt und
%  Version 1.3c oder später ist Teil aller Verteilungen von LaTeX 2005/12/01
%  oder später und dieses Werks. Dieses Werk hat den LPPL-Verwaltungs-Status
%  "author-maintained", wird somit allein durch den Autor verwaltet. Der
%  aktuelle Verwalter und Autor dieses Werkes ist Falk Hanisch.
%
% ----------------------------------------------------------------------------
%
% \fi
%
% \CharacterTable
%  {Upper-case    \A\B\C\D\E\F\G\H\I\J\K\L\M\N\O\P\Q\R\S\T\U\V\W\X\Y\Z
%   Lower-case    \a\b\c\d\e\f\g\h\i\j\k\l\m\n\o\p\q\r\s\t\u\v\w\x\y\z
%   Digits        \0\1\2\3\4\5\6\7\8\9
%   Exclamation   \!     Double quote  \"     Hash (number) \#
%   Dollar        \$     Percent       \%     Ampersand     \&
%   Acute accent  \'     Left paren    \(     Right paren   \)
%   Asterisk      \*     Plus          \+     Comma         \,
%   Minus         \-     Point         \.     Solidus       \/
%   Colon         \:     Semicolon     \;     Less than     \<
%   Equals        \=     Greater than  \>     Question mark \?
%   Commercial at \@     Left bracket  \[     Backslash     \\
%   Right bracket \]     Circumflex    \^     Underscore    \_
%   Grave accent  \`     Left brace    \{     Vertical bar  \|
%   Right brace   \}     Tilde         \~}
%
% \iffalse
%%% From File: tudscr-color.dtx
%<*dtx>
% \fi
%
\ifx\ProvidesFile\undefined\def\ProvidesFile#1[#2]{}\fi
\ProvidesFile{tudscr-color.dtx}[2019/06/28 v2.06 TUD-Script\space%
%
% \iffalse
%</dtx>
%<package>\ProvidesPackage{tudscrcolor}[%
%<*package>
%!TUD@Version
%</package>
%<package>  package
%<*dtx|package>
% \fi
  (corporate design colors)%
]
% \iffalse
%</dtx|package>
%<*dtx>
\documentclass[english,ngerman,xindy]{tudscrdoc}
\iftutex
  \usepackage{fontspec}
\else
  \usepackage[T1]{fontenc}
  \usepackage[ngerman=ngerman-x-latest]{hyphsubst}
\fi
\usepackage{babel}
\usepackage{tudscrfonts}
\KOMAoptions{parskip=half-}
\usepackage{bookmark}
\usepackage[babel]{microtype}

\CodelineIndex
\RecordChanges
\GetFileInfo{tudscr-color.dtx}
\title{\file{\filename}}
\author{Falk Hanisch\qquad\expandafter\mailto\expandafter{\tudscrmail}}
\date{\fileversion\nobreakspace(\filedate)}

\begin{document}
  \maketitle
  \tableofcontents
  \DocInput{\filename}
\end{document}
%</dtx>
% \fi
%
% \selectlanguage{ngerman}
%
% \section{Das Paket \pkg{tudscrcolor} -- Die Farben des \CDs}
%
% Das \TUDCD legt nicht nur die zu nutzenden Schriften und das Layout sondern 
% auch die zu verwendenden Farben fest. Diese werden nachfolgend für das CMYK"~ 
% und RGB"~Farbmodel definiert. Sie können im Dokument mit sämtlichen Befehlen
% zur Farbauswahl wie \cs{color}\marg{Farbe} oder \cs{textcolor}\marg{Farbe} 
% verwendet werden.
%
% \StopEventually{\PrintIndex\PrintChanges\PrintToDos}
%
% \iffalse
%<*package>
% \fi
%
% \subsection{Optionen und Befehle}
%
% \begin{option}{newcolors}
% \begin{option}{reduced}
% \begin{option}{oldcolors}
% \begin{option}{full}
% Das Paket kann entweder mit einfachen, für \TUDScript ausreichenden oder 
% erweiterten Farbdefinitionen geladen werden. Letztere definieren zusätzliche 
% Farbbezeichnungen für die Kompatibilität zu alten TUD-Klassen.
%    \begin{macrocode}
\DeclareOption{newcolors}{\let\tud@setcolors@add\relax}
\DeclareOption{reduced}{\let\tud@setcolors@add\relax}
\DeclareOption{oldcolors}{\let\tud@setcolors@add\tud@setcolors@old}
\DeclareOption{full}{\let\tud@setcolors@add\tud@setcolors@old}
%    \end{macrocode}
% \end{option}^^A full
% \end{option}^^A oldcolors
% \end{option}^^A reduced
% \end{option}^^A newcolors
% \begin{macro}{\setcdcolors}
% Der Befehl \cs{setcdcolors} definiert die Farben des \CDs. Das Argument dient
% zur Auswahl des gewünschten Farbmodels. Dies kann dazu genutzt werden,
% innerhalb des Dokumentes die Definition der Farben für ein neues Farbmodell 
% zu ändern.
%    \begin{macrocode}
\newcommand*\setcdcolors[1]{%
  \selectcolormodel{#1}%
  \tud@setcolors@do%
}
%    \end{macrocode}
% \end{macro}^^A \setcdcolors
% \begin{macro}{\tud@color}
% \changes{v2.04}{2015/04/21}{neu}^^A
% Der Befehl wählt eine in einem Makro gespeicherte Farbe nur aus, wenn dieses 
% auch tatsächlich definiert ist. Dieser Befehl wird von den \TUDScript-Klassen 
% für die Umsetzung des Layouts verwendet.
%    \begin{macrocode}
\newcommand*\tud@color[1]{\ifdefvoid{#1}{}{\color{#1}}}
%    \end{macrocode}
% \end{macro}^^A \tud@color
%
% \subsection{Farbdefinitionen}
% \subsubsection{Notwendige Farben für \TUDScript}
%
% \begin{macro}{\tud@setcolors@do}
% Die Definitionen der Farben erfolgt erst durch die Ausführung von 
% \cs{tud@setcolors@do}. Damit wird es möglich, auf die angegebenen Optionen in
% Form von \cs{tud@setcolors@add} oder den durch das Paket \pkg{xcolor}
% angegebenen Farbraum zu reagieren.
%    \begin{macrocode}
\newcommand*\tud@setcolors@do{%
%    \end{macrocode}
% \begin{TUDcolor}{HKS41}
% Die primäre Hausfarbe (dunkles Blau)
%    \begin{macrocode}
  \definecolor{HKS41}{cmyk/RGB/rgb}{%
    1.00,0.70,0.10,0.50/011,042,081/0.0431372549,0.16470588235,0.31764705882%
  }%
%    \end{macrocode}
% \end{TUDcolor}^^A HKS41
% \begin{TUDcolor}{HKS92}
% Die sekundäre Hausfarbe (grau), allein und ausschließlich für die Verwendung
% in der Geschäftsausstattung und nicht für Fließtext, Grafiken etc.
%    \begin{macrocode}
  \definecolor{HKS92}{cmyk/RGB/rgb}{%
    0.10,0.00,0.05,0.65/080,089,085/0.31372549019,0.34901960784,0.33333333333%
  }%
%    \end{macrocode}
% \end{TUDcolor}^^A HKS92
% \begin{TUDcolor}{HKS44}
% Auszeichnungen 1. Kategorie (helles Blau)
%    \begin{macrocode}
  \definecolor{HKS44}{cmyk/RGB/rgb}{%
    1.00,0.50,0.00,0.00/000,089,163/0,0.34901960784,0.63921568627%
  }%
%    \end{macrocode}
% \end{TUDcolor}^^A HKS44
% \begin{TUDcolor}{HKS36}
% \begin{TUDcolor}{HKS33}
% \begin{TUDcolor}{HKS57}
% \begin{TUDcolor}{HKS65}
% Auszeichnungen 2. Kategorie (Indigo, Purpur, dunkles Grün, helles Grün)
%    \begin{macrocode}
  \definecolor{HKS36}{cmyk/RGB/rgb}{%
    0.80,0.90,0.00,0.00/081,041,127/0.31764705882,0.16078431372,0.49803921568%
  }%
  \definecolor{HKS33}{cmyk/RGB/rgb}{%
    0.50,1.00,0.00,0.00/129,026,120/0.50588235294,0.10196078431,0.47058823529%
  }%
  \definecolor{HKS57}{cmyk/RGB/rgb}{%
    1.00,0.00,0.90,0.20/000,122,071/0,0.47843137254,0.28235294117%
  }%
  \definecolor{HKS65}{cmyk/RGB/rgb}{%
    0.65,0.00,1.00,0.00/034,173,054/0.13333333333,0.67843137254,0.21176470588%
  }%
%    \end{macrocode}
% \end{TUDcolor}^^A HKS65
% \end{TUDcolor}^^A HKS57
% \end{TUDcolor}^^A HKS33
% \end{TUDcolor}^^A HKS36
% \begin{TUDcolor}{HKS07}
% Ausnahmefarbe (Orange)
%    \begin{macrocode}
  \definecolor{HKS07}{cmyk/RGB/rgb}{%
    0.00,0.60,1.00,0.00/232,123,020/0.90980392156,0.48235294117,0.07843137254%
  }%
%    \end{macrocode}
% \end{TUDcolor}^^A HKS07
% \begin{TUDcolor}{cddarkblue}
% \begin{TUDcolor}{cdgray}
% \begin{TUDcolor}{cdgrey}
% \begin{TUDcolor}{cdblue}
% \begin{TUDcolor}{cdindigo}
% \begin{TUDcolor}{cdpurple}
% \begin{TUDcolor}{cddarkgreen}
% \begin{TUDcolor}{cdgreen}
% \begin{TUDcolor}{cdorange}
% Die definierten Grundfarben werden zur einfacheren Verwendung im Dokument 
% noch einmal speziell benannt.
%    \begin{macrocode}
  \colorlet{cddarkblue}{HKS41}%
  \colorlet{cdgray}{HKS92}%
  \colorlet{cdgrey}{HKS92}%
  \colorlet{cdblue}{HKS44}%
  \colorlet{cdindigo}{HKS36}%
  \colorlet{cdpurple}{HKS33}%
  \colorlet{cddarkgreen}{HKS57}%
  \colorlet{cdgreen}{HKS65}%
  \colorlet{cdorange}{HKS07}%
%    \end{macrocode}
% \end{TUDcolor}^^A cdorange
% \end{TUDcolor}^^A cdgreen
% \end{TUDcolor}^^A cddarkgreen
% \end{TUDcolor}^^A cdpurple
% \end{TUDcolor}^^A cdindigo
% \end{TUDcolor}^^A cdblue
% \end{TUDcolor}^^A cdgrey
% \end{TUDcolor}^^A cdgray
% \end{TUDcolor}^^A cddarkblue
% Damit enden die notwendigen Farbdefinitionen für das \pkg{tudscrcolor}-Paket.
% Abhängig von den gewählten Optionen werden mit \cs{tud@setcolors@add} ggf.
% zusätzliche Farbnamen definiert.
%    \begin{macrocode}
  \tud@setcolors@add%
}
%    \end{macrocode}
% \end{macro}^^A \tud@setcolors@do
%
% \subsubsection{Zusätzliche Farben für alte TUD-Klassen}
%
% \begin{macro}{\tud@setcolors@add}
% \begin{macro}{\tud@setcolors@old}
% Die erweiterten Farbbefehle werden durch \pkg{tudscrcolor} definiert, wenn
% das Paket explizit mit der Option \opt{oldcolors} geladen wird. Damit werden 
% alle gängigen Farbdefinitionen der vielen Insellösungen des \LaTeX-Universums
% an der \TnUD unterstützt.
%    \begin{macrocode}
\newcommand*\tud@setcolors@add{}
\newcommand*\tud@setcolors@old{%
  \colorlet{HKS41K10}{HKS41!10}%
  \colorlet{HKS41K20}{HKS41!20}%
  \colorlet{HKS41K30}{HKS41!30}%
  \colorlet{HKS41K40}{HKS41!40}%
  \colorlet{HKS41K50}{HKS41!50}%
  \colorlet{HKS41K60}{HKS41!60}%
  \colorlet{HKS41K70}{HKS41!70}%
  \colorlet{HKS41K80}{HKS41!80}%
  \colorlet{HKS41K90}{HKS41!90}%
  \colorlet{HKS41K100}{HKS41!100}%
  \colorlet{HKS92K10}{HKS92!10}%
  \colorlet{HKS92K20}{HKS92!20}%
  \colorlet{HKS92K30}{HKS92!30}%
  \colorlet{HKS92K40}{HKS92!40}%
  \colorlet{HKS92K50}{HKS92!50}%
  \colorlet{HKS92K60}{HKS92!60}%
  \colorlet{HKS92K70}{HKS92!70}%
  \colorlet{HKS92K80}{HKS92!80}%
  \colorlet{HKS92K90}{HKS92!90}%
  \colorlet{HKS92K100}{HKS92!100}%
  \colorlet{HKS44K10}{HKS44!10}%
  \colorlet{HKS44K20}{HKS44!20}%
  \colorlet{HKS44K30}{HKS44!30}%
  \colorlet{HKS44K40}{HKS44!40}%
  \colorlet{HKS44K50}{HKS44!50}%
  \colorlet{HKS44K60}{HKS44!60}%
  \colorlet{HKS44K70}{HKS44!70}%
  \colorlet{HKS44K80}{HKS44!80}%
  \colorlet{HKS44K90}{HKS44!90}%
  \colorlet{HKS44K100}{HKS44!100}%
  \colorlet{HKS36K10}{HKS36!10}%
  \colorlet{HKS36K20}{HKS36!20}%
  \colorlet{HKS36K30}{HKS36!30}%
  \colorlet{HKS36K40}{HKS36!40}%
  \colorlet{HKS36K50}{HKS36!50}%
  \colorlet{HKS36K60}{HKS36!60}%
  \colorlet{HKS36K70}{HKS36!70}%
  \colorlet{HKS36K80}{HKS36!80}%
  \colorlet{HKS36K90}{HKS36!90}%
  \colorlet{HKS36K100}{HKS36!100}%
  \colorlet{HKS33K10}{HKS33!10}%
  \colorlet{HKS33K20}{HKS33!20}%
  \colorlet{HKS33K30}{HKS33!30}%
  \colorlet{HKS33K40}{HKS33!40}%
  \colorlet{HKS33K50}{HKS33!50}%
  \colorlet{HKS33K60}{HKS33!60}%
  \colorlet{HKS33K70}{HKS33!70}%
  \colorlet{HKS33K80}{HKS33!80}%
  \colorlet{HKS33K90}{HKS33!90}%
  \colorlet{HKS33K100}{HKS33!100}%
  \colorlet{HKS57K10}{HKS57!10}%
  \colorlet{HKS57K20}{HKS57!20}%
  \colorlet{HKS57K30}{HKS57!30}%
  \colorlet{HKS57K40}{HKS57!40}%
  \colorlet{HKS57K50}{HKS57!50}%
  \colorlet{HKS57K60}{HKS57!60}%
  \colorlet{HKS57K70}{HKS57!70}%
  \colorlet{HKS57K80}{HKS57!80}%
  \colorlet{HKS57K90}{HKS57!90}%
  \colorlet{HKS57K100}{HKS57!100}%
  \colorlet{HKS65K10}{HKS65!10}%
  \colorlet{HKS65K20}{HKS65!20}%
  \colorlet{HKS65K30}{HKS65!30}%
  \colorlet{HKS65K40}{HKS65!40}%
  \colorlet{HKS65K50}{HKS65!50}%
  \colorlet{HKS65K60}{HKS65!60}%
  \colorlet{HKS65K70}{HKS65!70}%
  \colorlet{HKS65K80}{HKS65!80}%
  \colorlet{HKS65K90}{HKS65!90}%
  \colorlet{HKS65K100}{HKS65!100}%
  \colorlet{HKS07K10}{HKS07!10}%
  \colorlet{HKS07K20}{HKS07!20}%
  \colorlet{HKS07K30}{HKS07!30}%
  \colorlet{HKS07K40}{HKS07!40}%
  \colorlet{HKS07K50}{HKS07!50}%
  \colorlet{HKS07K60}{HKS07!60}%
  \colorlet{HKS07K70}{HKS07!70}%
  \colorlet{HKS07K80}{HKS07!80}%
  \colorlet{HKS07K90}{HKS07!90}%
  \colorlet{HKS07K100}{HKS07!100}%
  \colorlet{HKS41-10}{HKS41!10}%
  \colorlet{HKS41-20}{HKS41!20}%
  \colorlet{HKS41-30}{HKS41!30}%
  \colorlet{HKS41-40}{HKS41!40}%
  \colorlet{HKS41-50}{HKS41!50}%
  \colorlet{HKS41-60}{HKS41!60}%
  \colorlet{HKS41-70}{HKS41!70}%
  \colorlet{HKS41-80}{HKS41!80}%
  \colorlet{HKS41-90}{HKS41!90}%
  \colorlet{HKS41-100}{HKS41!100}%
  \colorlet{HKS92-10}{HKS92!10}%
  \colorlet{HKS92-20}{HKS92!20}%
  \colorlet{HKS92-30}{HKS92!30}%
  \colorlet{HKS92-40}{HKS92!40}%
  \colorlet{HKS92-50}{HKS92!50}%
  \colorlet{HKS92-60}{HKS92!60}%
  \colorlet{HKS92-70}{HKS92!70}%
  \colorlet{HKS92-80}{HKS92!80}%
  \colorlet{HKS92-90}{HKS92!90}%
  \colorlet{HKS92-100}{HKS92!100}%
  \colorlet{HKS44-10}{HKS44!10}%
  \colorlet{HKS44-20}{HKS44!20}%
  \colorlet{HKS44-30}{HKS44!30}%
  \colorlet{HKS44-40}{HKS44!40}%
  \colorlet{HKS44-50}{HKS44!50}%
  \colorlet{HKS44-60}{HKS44!60}%
  \colorlet{HKS44-70}{HKS44!70}%
  \colorlet{HKS44-80}{HKS44!80}%
  \colorlet{HKS44-90}{HKS44!90}%
  \colorlet{HKS44-100}{HKS44!100}%
  \colorlet{HKS36-10}{HKS36!10}%
  \colorlet{HKS36-20}{HKS36!20}%
  \colorlet{HKS36-30}{HKS36!30}%
  \colorlet{HKS36-40}{HKS36!40}%
  \colorlet{HKS36-50}{HKS36!50}%
  \colorlet{HKS36-60}{HKS36!60}%
  \colorlet{HKS36-70}{HKS36!70}%
  \colorlet{HKS36-80}{HKS36!80}%
  \colorlet{HKS36-90}{HKS36!90}%
  \colorlet{HKS36-100}{HKS36!100}%
  \colorlet{HKS33-10}{HKS33!10}%
  \colorlet{HKS33-20}{HKS33!20}%
  \colorlet{HKS33-30}{HKS33!30}%
  \colorlet{HKS33-40}{HKS33!40}%
  \colorlet{HKS33-50}{HKS33!50}%
  \colorlet{HKS33-60}{HKS33!60}%
  \colorlet{HKS33-70}{HKS33!70}%
  \colorlet{HKS33-80}{HKS33!80}%
  \colorlet{HKS33-90}{HKS33!90}%
  \colorlet{HKS33-100}{HKS33!100}%
  \colorlet{HKS57-10}{HKS57!10}%
  \colorlet{HKS57-20}{HKS57!20}%
  \colorlet{HKS57-30}{HKS57!30}%
  \colorlet{HKS57-40}{HKS57!40}%
  \colorlet{HKS57-50}{HKS57!50}%
  \colorlet{HKS57-60}{HKS57!60}%
  \colorlet{HKS57-70}{HKS57!70}%
  \colorlet{HKS57-80}{HKS57!80}%
  \colorlet{HKS57-90}{HKS57!90}%
  \colorlet{HKS57-100}{HKS57!100}%
  \colorlet{HKS65-10}{HKS65!10}%
  \colorlet{HKS65-20}{HKS65!20}%
  \colorlet{HKS65-30}{HKS65!30}%
  \colorlet{HKS65-40}{HKS65!40}%
  \colorlet{HKS65-50}{HKS65!50}%
  \colorlet{HKS65-60}{HKS65!60}%
  \colorlet{HKS65-70}{HKS65!70}%
  \colorlet{HKS65-80}{HKS65!80}%
  \colorlet{HKS65-90}{HKS65!90}%
  \colorlet{HKS65-100}{HKS65!100}%
  \colorlet{HKS07-10}{HKS07!10}%
  \colorlet{HKS07-20}{HKS07!20}%
  \colorlet{HKS07-30}{HKS07!30}%
  \colorlet{HKS07-40}{HKS07!40}%
  \colorlet{HKS07-50}{HKS07!50}%
  \colorlet{HKS07-60}{HKS07!60}%
  \colorlet{HKS07-70}{HKS07!70}%
  \colorlet{HKS07-80}{HKS07!80}%
  \colorlet{HKS07-90}{HKS07!90}%
  \colorlet{HKS07-100}{HKS07!100}%
}
%    \end{macrocode}
% \end{macro}^^A \tud@setcolors@old
% \end{macro}^^A \tud@setcolors@add
%
% \subsection{Ausführung der Optionen}
%
% Zum Schluss werden die Optionen ausgeführt und ggf. an \pkg{xcolor} 
% weitergereicht. Anschließend werden die Farben für das Dokument definiert.
% Ohne die Angabe eines optionalen Argumentes an das Paket \pkg{xcolor} erfolgt
% die Definition für den gewählten bzw. standardmäßig eingestellten Farbraum.
%    \begin{macrocode}
\DeclareOption*{\PassOptionsToPackage{\CurrentOption}{xcolor}}
\ExecuteOptions{reduced}
\ProcessOptions\relax
\RequirePackage{xcolor}[2007/01/21]
\tud@setcolors@do%
%    \end{macrocode}
%
% \iffalse
%</package>
% \fi
%
% \Finale
%
\endinput


\documentclass[ngerman,cdfont=false]{tudscrreprt}
\usepackage[T1]{fontenc}
\ifpdftex{\usepackage[ngerman=ngerman-x-latest]{hyphsubst}}{}
\usepackage{babel}
\usepackage[math]{blindtext}
% bessere Schrifen
\usepackage{lmodern}
\begin{document}
\faculty{Juristische Fakultät}
\department{Fachrichtung Strafrecht}
\institute{Institut für Kriminologie}
\chair{Lehrstuhl für Kriminalprognose}
\date{18.02.2015}
\author{Mickey Mouse}
\title{Der Geldspeicher in Entenhausen}
\maketitle
\blinddocument
\end{document}


\documentclass[english]{tudscrreprt}
\usepackage[english]{babel}
\usepackage[T1]{fontenc}
\usepackage{tudscrsupervisor}
\AfterPackage*{hyperref}{%
\usepackage[%
acronym,% Abkürzungen
symbols,% Formelzeichen
nomain,% kein Glossar
nogroupskip,%
toc,%
section=chapter,%
nostyles,%
translate=babel,%
% mit Tex Live einfach verwendbar
xindy={language=english},
]{glossaries}
\makeglossaries
} % Ende von AfterPackage

\AfterPackage*{glossaries}{%
   \newglossarystyle{acrotabu}{%
   \renewenvironment{theglossary}{%
      \begin{tabu}{@{}lX<{\strut}l@{}}%
      }{%
      \end{tabu}\par\bigskip%
      }%
      \renewcommand*{\glossaryheader}{}%
      \renewcommand*{\glsgroupheading}[1]{}%
      \renewcommand*{\glsgroupskip}{}%
      \renewcommand*{\glossentry}[2]{%
      \glsentryitem{##1}% Entry number if required
      \glstarget{##1}{\sffamily\bfseries\glossentryname{##1}} &
      \glsentrydesc{##1} &
      ##2\tabularnewline
   }
}

\newcommand*{\newsymbol}[5][]{%
   \newglossaryentry{#2}{%
      type=symbols,%
      name={#3},%
      description={\nopostdesc},%
      symbol={\ensuremath{#4}},%
      user1={\ensuremath{\mathrm{#5}}},%
      sort={#2},%
      #1%
   }%
}

\defglsentryfmt[symbols]{%
      \ifmmode%
   \glssymbol{\glslabel}%
      \else%
   \glsgenentryfmt~\glsentrysymbol{\glslabel}%
      \fi%
}

\newglossarystyle{symblongtabu}{%
   \renewenvironment{theglossary}{%
   \begin{longtabu}[l]{ccX<{\strut}l}%
   }{%
   \end{longtabu}%
   }%
      \renewcommand*{\glsgroupheading}[1]{}%
      \renewcommand*{\glsgroupskip}{}%
      \renewcommand*{\glossaryheader}{%
         \toprule
         \bfseries Formelzeichen & \bfseries Einheit &
         \bfseries Bezeichnung
         \tabularnewline\midrule\endhead%
         \bottomrule\endfoot%
      }%
      \renewcommand*{\glossentry}[2]{%
         \glsentryitem{##1}% Entry number if required
         \glstarget{##1}{\glossentrysymbol{##1}} &
         \glsentryuseri{##1} &
         \glossentryname{##1}
         \tabularnewline%
      }%
   }
} % Ende von AfterPackage

\usepackage{array}
\usepackage{tabularx}
\usepackage{tabulary}
\usepackage{tabu}
\usepackage{longtable}
\usepackage{booktabs}
\usepackage[babel]{microtype}
\usepackage{xfrac}
\usepackage{ellipsis}
\let\ellipsispunctuation\relax
\usepackage[colorlinks,linkcolor=blue]{hyperref}
\begin{document}
% Kompilieren läuft problemlos durch
\printacronyms[style=acrotabu] % Ausgabe 
%Abkürzungsverz.

% Sobald ich den selbst definierten Stil nicht verwende, funktioniert alles.
\printsymbols[style=symblongtabu] % Ausgabe Symbolverzeichnis

\tableofcontents

\newacronym{apsp}{APSP}{All-Pairs Shortest Path}
\newacronym{spsp}{SPSP}{Single-Pair Shortest Path}
\newacronym{sssp}{SSSP}{Single-Source Shortest Path}

\newsymbol{m}{Masse}{m}{kg}

\chapter{Einleitung}

\section{Die Verwendung von Akronymen und Symbolen}


\section{Textteil}
In der Graphentheorie wird häufig die Lösung des Problems des kürzesten
Pfades zwischen zwei Knoten gesucht. Dieses Problem wird häufig auch
mit \gls{spsp} bezeichnet. Es lässt sich auf die Variationen \gls{sssp}
und \gls{apsp} erweitern. Für die Lösung von \gls{spsp}, \gls{sssp}
oder \gls{apsp} kommen unterschiedliche Algorithmen zum Einsatz.

Die \gls{m} ist eine SI-Einheit.

\end{document}
